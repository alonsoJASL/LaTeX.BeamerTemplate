% -------- CNU Public License v3 --------
%
% This BEAMER TEMPLATE is under GNU General Public License v.3
% check it our here: http://www.gnu.org/licenses/gpl-3.0.txt
%
% Also, I got this from a template available online, but I don't recall nor I have found 
% the original author's name. I thank him for helping me when I needed, and be sure
% that whenever I find his original template, I'll give him the proper recognition.
% 
% Finally, this template includes some Spanish, but only for the things relevant to people
% who want to present in Spanish. 
% --

\documentclass{beamer}

% This file is a solution template for:
% - Talk at a conference/colloquium.
% - Talk length is about 20min.
% - Style is ornate.

\mode<presentation>
{
  \usetheme{Dresden} % or Goettingen, Berkeley, ...

% -- Some colours and theme tweaks: 
%\usecolortheme[RGB={0,107,84}]{structure} % Dark (ITAM) GREEN
%\usecolortheme[RGB={124,33,30}]{structure} % Nice RED
\usecolortheme[RGB={55,96,146}]{structure} % Strong BLUE (spanish joke)
\useinnertheme{circles}
\usefonttheme[onlylarge]{structuresmallcapsserif}
\usefonttheme[onlysmall]{structurebold}
% --  
  
  \setbeamercovered{transparent}
  % or whatever (possibly just delete it)
}

\usepackage[utf8]{inputenc}  
%\usepackage[LGR,T1]{fontenc}    

% -- ¿Odia los acentos (e.g. ) en LaTeX? con lo siguiente puede escribir normal, y LaTeX 
% no llorará por escribir "llorará" y no "llorar\'a".
\usepackage[greek,spanish]{babel}
% --
\usepackage{times}
\usepackage{amsmath}
\usepackage{amssymb}
\usepackage{verbatim}
\usepackage{schemabloc}
\usepackage{latexsym}
\usepackage{hyperref}
\usepackage{IEEEtrantools}
\usepackage{subfigure}
\usepackage{wrapfig}

\graphicspath{{./images}} % just add more IMAGES folders here.

\newcommand{\autor}{AUTHOR NAME HERE}
\newcommand{\titulo}{TITLE HERE}
\newcommand{\subtitulo}{Include Only If Paper Has a Subtitle}

\title[Short Paper Title] % (optional, use only with long paper titles)
{\titulo}

\subtitle{\subtitulo}

\author[] % (optional, use only with lots of authors)
{\autor\inst{1} } %\and S.~Another\inst{2}

% - Give the names in the same order as the appear in the paper.
% - Use the \inst{?} command only if the authors have different
%   affiliation.
\institute[] % (optional, but mostly needed)
{
  \inst{1}%
  Department of Something\\
  University of Great Creations and Magic}
 % \and
  %\inst{2}
  
% - Use the \inst command only if there are several affiliations.
% - Keep it simple, no one is interested in your street address.

\date[] % (optional, should be abbreviation of conference name)
{DATE HERE}
% - Either use conference name or its abbreviation.
% - Not really informative to the audience, more for people (including
%   yourself) who are reading the slides online

\subject{} % This is only inserted into the PDF information catalog.

% -- LOGOS
% If you have a file called "university-logo-filename.xxx", where xxx
% is a graphic format that can be processed by latex or pdflatex,
% resp., then you can add a logo as follows:

% \pgfdeclareimage[height=0.5cm]{university-logo}{university-logo-filename}
% \logo{\pgfuseimage{university-logo}}
% --

% Delete this, if you do not want the table of contents to pop up at
% the beginning of each subsection:
\AtBeginSubsection[]
{
  \begin{frame}<beamer>{Outline}
    \tableofcontents[currentsection,currentsubsection]
  \end{frame}
}

% If you wish to uncover everything in a step-wise fashion, uncomment
% the following command: 

%\beamerdefaultoverlayspecification{<+->}

\begin{document}
\begin{frame}
  \titlepage 
  \end{frame}

\begin{frame}{Outline}
  \tableofcontents
  % You might wish to add the option [pausesections]
\end{frame}

% -- Heartfelt pep talk on speeches, thanks original author, this actually works!
% Structuring a talk is a difficult task and the following structure
% may not be suitable. Here are some rules that apply for this
% solution: 

% - Exactly two or three sections (other than the summary).
% - At *most* three subsections per section.
% - Talk about 30s to 2min per frame. So there should be between about
%   15 and 30 frames, all told.

% - A conference audience is likely to know very little of what you
%   are going to talk about. So *simplify*!
% - In a 20min talk, getting the main ideas across is hard
%   enough. Leave out details, even if it means being less precise than
%   you think necessary.
% - If you omit details that are vital to the proof/implementation,
%   just say so once. Everybody will be happy with that.

% My two cents: Even though we partly use Beamer to do formulae and math, we must 
% remember that in presentations math is like sex: it's great to have it, just not in front of
% everybody ;)
% 
% --

\section{Motivation} 
\subsection{The Basic Problem That We Studied}

\begin{frame}{Make Titles Informative. Use Uppercase Letters.}{Subtitles are optional.}
  % - A title should summarize the slide in an understandable fashion
  %   for anyone how does not follow everything on the slide itself.

  \begin{itemize}
  \item
    Use \texttt{itemize} a lot.
  \item
    Use very short sentences or short phrases.
  \end{itemize}
\end{frame}

\begin{frame}{Make Titles Informative.}

  You can create overlays\dots
  \begin{itemize}
  \item using the \texttt{pause} command:
    \begin{itemize}
    \item
      First item.
      \pause
    \item    
      Second item.
    \end{itemize}
  \item
    using overlay specifications:
    \begin{itemize}
    \item<3->
      First item.
    \item<4->
      Second item.
    \end{itemize}
  \item
    using the general \texttt{uncover} command:
    \begin{itemize}
      \uncover<5->{\item
        First item.}
      \uncover<6->{\item
        Second item.}
    \end{itemize}
  \end{itemize}
\end{frame}


\subsection{Previous Work}

\begin{frame}{Make Titles Informative.}
\end{frame}
\begin{frame}{Make Titles Informative.}
\end{frame}



\section{Our Results/Contribution}

\subsection{Main Results}

\begin{frame}{Make Titles Informative.}
\end{frame}
\begin{frame}{Make Titles Informative.}
\end{frame}
\begin{frame}{Make Titles Informative.}
\end{frame}


\subsection{Basic Ideas for Proofs/Implementation}

\begin{frame}{Make Titles Informative.}
\end{frame}
\begin{frame}{Make Titles Informative.}
\end{frame}
\begin{frame}{Make Titles Informative.}
\end{frame}

\section*{Summary}

\begin{frame}{Summary}

  % Keep the summary VERY short.
  \begin{itemize}
  \item
    The \alert{first main message} of your talk in one or two lines.
  \item
    The \alert{second main message} of your talk in one or two lines.
  \item
    Perhaps a \alert{third message}, but not more than that.
  \end{itemize}
  
  % The following outlook is optional.. and super cool.
  \vskip0pt plus.5fill
  \begin{itemize}
  \item
    Outlook
    \begin{itemize}
    \item
      Something you haven't solved.
    \item
      Something else you haven't solved.
    \end{itemize}
  \end{itemize}
\end{frame}

% All of the following is optional and typically not needed. 
\appendix
\section<presentation>*{\appendixname}
\subsection<presentation>*{For Further Reading}

\begin{frame}[allowframebreaks]
  \frametitle<presentation>{For Further Reading}
    
  \begin{thebibliography}{10}
    
  \beamertemplatebookbibitems
  % Start with overview books.

  \bibitem{Author1990}
    A.~Author.
    \newblock {\em Handbook of Everything}.
    \newblock Some Press, 1990.
 
    
  \beamertemplatearticlebibitems
  % Followed by interesting articles. Keep the list short. 

  \bibitem{Someone2000}
    S.~Someone.
    \newblock On this and that.
    \newblock {\em Journal of This and That}, 2(1):50--100,
    2000.
  \end{thebibliography}
\end{frame}

\end{document}


